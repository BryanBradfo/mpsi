\documentclass[a4paper,10pt]{article}

\usepackage[utf8]{inputenc}
\usepackage[french]{babel}
\usepackage[T1]{fontenc}
\usepackage{amsmath}
\usepackage{amsfonts}
\usepackage{amssymb}
\usepackage{graphicx}
\usepackage{tikz}
\begin{document}
$\tikz[baseline=(letter.base)]\node[draw,circle,inner sep=1pt] (letter) {16};\  h\mapsto (1+h)^{\frac{1}{h}}\ est\ d\acute{e}finie\ sur\ I=]-1,0[\cup]0, +\infty[.\ Elle\ est\ de\ classe\ C^{\infty}\ donc\ admet\ un\ DL\ de\ tout\ ordre\ en\  particulier\ d'ordre\ 3\
En\ effet,\
(1+h)^{\frac{1}{h}} = e^{\frac{1}{h}ln(1+h)} = e^{1-\frac{h}{2}+\frac{h^{2}}{3}+o(h^{2})}\
= e^1.e^{\frac{-h}{2}}.e^{\frac{h^3}{3}+o(h^2)}\
= e(1-\frac{h}{2}+\frac{h^2}{8}+o(h^2))(1+\frac{h^2}{3}+o(h^2))\
= (e-\frac{eh}{2}+\frac{eh^2}{8}+o(h^2))(1+\frac{h^2}{3}+o(h^2))\
= e+\frac{eh^2}{3}-\frac{eh}{2}+\frac{eh^2}{8}+o(h^2)
= e-\frac{eh}{2}+\frac{11}{24}h^2 e+o(h^2)h\to_0\
$
\begin{center}
$(1+h)^{\frac{1}{h}}=e-\frac{eh}{2}+\frac{11}{24}h^e+o(h^2)h\to_0$
\end{center}
\end{document}
